\documentclass[10pt,a4paper]{article}
\usepackage[utf8]{inputenc}
\usepackage[french]{babel}
\usepackage[T1]{fontenc}
\usepackage{amsmath}
\usepackage{amsfonts}
\usepackage{amssymb}
\usepackage{graphicx}
\begin{document}
\subsection{Algorithme implémenté}

Nous avons implémenté un solveur SAT [1] basé sur DPLL. La procédure de base
consiste à maintenir une interprétation partielle et une formule en forme CNF.
On applique les contraintes générées par l'interprétation et on suppose vraies
les clauses à un seul littéral. Lorsqu'il ne reste plus de contrainte forcée,
on suppose vrai un littéral et on poursuit; en cas d'échec, on suppose on le
littéral faux et on recommence.

Les deux optimisations sont :

\paragraph{Retour en arrière non chronologique}

Si supposer un littéral $l$ conduit à un échec, on inspecte les littéraux
responsables de cet échec. Si $l$ n'est pas l'un d'eux, il est inutile de
recommencer en supposant $l$ faux.

\paragraph{Apprentissage}

On garde en mémoire tous les conflits causés par les échecs précédents. Lorsque
l'on essaie une nouvelle branche suite à un échec, on ajoute ces conflits à la
liste des clauses à satisfaire.

\subsection{Choix d'implémentation} Nous avons utilisé des structures de
données purement fonctionnelles et nous manipulons les formules et clauses de
façon assez naïve (et inefficace, mais les résultats sont tous de même
satisfaisants). Les formules sont des listes de listes de couples {\tt int * bool}
(dans la version de base). Les types sont un peu enrichis par les deux
optimisations mentionnées ci-dessus.

On remarquera que chaque propagation de contrainte nécessite deux parcours
linéaires des clauses en cours de considération.

\subsection{Résultats} La version de base de DPLL résoud un Sudoku en ~1.4s
environ. Les deux optimisations offrent une légère accélération ($< 10\%$),
mais lorsqu'il est peu utile, l'apprentissage ralentit le calcul ($< 10\%$
aussi). L'usage du backjumping était rare, probablement parce que les formules 
caractéristiques d'un sudoku permettent peu d'exploiter ces optimisations. Par
exemple, lors d'un test, nous avons constaté qu'un grille de Sudoku nécessitait
57 choix de littéraux et que le retour en arrière non chronologique était
appliqué à seulement 3 de ces choix.

On a aussi essayé une version impérative, en général plus rapide d'environ
$30\%$ mais parfois deux fois plus lente.

On pourrait vouloir améliorer les versions purement fonctionnelles en plaçant
les clauses dans un \emph{file de priorité}, afin de pouvoir rapidement
extraire les clauses qui n'ont plus qu'un seul littéral.


\subsection{Branches Git}

Nous avons conservé différentes versions grâce à des branches {\tt git}, visibles grâce
à la commande {\tt git branch}. Vous pouvez consulter le code d'un tag grâce à la
commande {\tt git checkout <branch>}.

Seules les branches ci-dessous ont un intérêt, les autres représentent des essais ratés :

\paragraph{master} DPLL avec retour en arrière non-chronologique.
\paragraph{base} DPLL de base
\paragraph{learning} DPLL avec retour en arrière non-chronologique et apprentissage
\paragraph{imperative} DPLL de base, version impérative

\end{document}
