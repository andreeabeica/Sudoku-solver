\documentclass[10pt,a4paper]{article}
\usepackage[utf8]{inputenc}
\usepackage[french]{babel}
\usepackage[T1]{fontenc}
\usepackage{amsmath}
\usepackage{amsfonts}
\usepackage{amssymb}
\usepackage{graphicx}
\begin{document}
\subsection{Algorithme implémenté}

Nous avons implémenté un solveur SAT [1] basé sur DPLL. La procédure de base
consiste à maintenir une interprétation partielle et une formule en forme CNF.
On applique les contraintes générées par l'interprétation et on suppose vraies
les clauses à un seul littéral. Lorsqu'il ne reste plus de contrainte forcée,
on suppose vrai un littéral et on poursuit; en cas d'échec, on suppose on le
littéral faux et on recommence.

Les deux optimisations sont :

\paragraph{Retour en arrière non chronologique}

Si supposer un littéral $l$ conduit à un échec, on inspecte les littéraux
responsables de cet échec. Si $l$ n'est pas l'un d'eux, il est inutile de
recommencer en supposant $l$ faux.

\paragraph{Apprentissage}

On garde en mémoire tous les conflits causés par les échecs précédents. Lorsque
l'on essaie une nouvelle branche suite à un échec, on ajoute ces conflits à la
liste des clauses à satisfaire.

\subsection{Choix d'implémentation} Nous avons utilisé des structures de
données purement fonctionnelles et nous manipulons les formules et clauses de
façon assez naïve (et inefficace, mais les résultats sont tous de même
satisfaisants). Les formules sont des listes de listes de couples {\tt int * bool}
(dans la version de base). Les types sont un peu enrichis par les deux
optimisations mentionnées ci-dessus.

On remarquera que chaque propagation de contrainte nécessite deux parcours
linéaires des clauses en cours de considération.

\subsection{Résultats} La version de base de DPLL résoud un Sudoku en ~1.5s
environ. Les deux optimisations provoquent chacune un ralentissement de 15\%,
probablement parce que les formules caractéristiques d'un sudoku permettent peu
d'exploiter ces optimisations. Par exemple, lors d'un test, nous avons constaté
qu'un grille de Sudoku nécessitait 57 choix de littéraux et que le retour en
arrière non chronologique était appliqué à seulement 3 de ces choix.

On peut descendre à ~1.1s en appliquant les deux optimisations suivantes:
\begin{enumerate} \item Lors de la propagation de contraintes générées par la
      supposition d'un littéral, toute clause qui n'a plus qu'un seul littéral
      est immédiatement supprimée et le littéral est placé dans une file
      d'attente. La version naïve se contentait d'appliquer les contraintes
      puis de répéter l'opération de recherche de clauses réduites à un seul
      littéral.  \item Toujours lors de la propagation de contraintes, une
    clause vide déclenche un échec immédiat. La version naïve se contentait de
rechercher des clauses vides après chaque propagation de contrainte.
\end{enumerate}

Ces deux optimisations n'ont pas été immédiatement considérées parce qu'elles
rendent le code moins lisible.

\subsection{Tags Git}

Nous avons conservé différentes versions grâce à des tags {\tt git}, visibles grâce
à la commande {\tt git tag}. Vous pouvez consulter le code d'un tag grâce à la
commande {\tt git checkout <tag>}.

Liste des tags : 

\paragraph{ds-trivial-lists} DPLL de base, aucune optimisation 
\paragraph{ds-lone-literals-lists} DPLL de base, les littéraux seuls sont automatiquement
ajoutés à une file d'attente.
\paragraph{ds-abort-on-empty-disjs-lists} DPLL de base,
optimisation précédente + échec immédiat lorsque l'on vide une clause.
\paragraph{algo-backtracking} Retour en arrière non chronologique.
\paragraph{algo-backtracking-ds-trivial-lists,\\ algo-backtracking-ds-abort-on-empty-disj} Retour en arrière non chronologique + les optimisations mentionnées ci-dessus.
\paragraph{algo-learning} Retour en arrière non chronologique + Apprentissage.

\end{document}
