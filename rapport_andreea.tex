\documentclass[10pt,a4paper]{article}
\usepackage[utf8]{inputenc}
\usepackage[french]{babel}
\usepackage[T1]{fontenc}
\usepackage{amsmath}
\usepackage{amsfonts}
\usepackage{amssymb}
\usepackage{graphicx}
\begin{document}
\title{Solving the Sudoku problem}
\author{Andreea Beica%
  \thanks{Electronic address: \texttt{beica@clipper.ens.fr}; }, Adrien Husson%
  \thanks{Electronic address: \texttt{adhusson@gmail.com}}}
\maketitle

\section{Encoder une grille Sudoku}

Comme d\'{e}crit dans l'article de Tjark Weber, la premi\`{e}re \'{e}tape du projet consiste \'{a} encoder les contraintes du jeu de Sudoku dans des formules en forme normale conjoctive.

Ainsi, une grille de Sudoku peut \^{e}tre caracteris\'{e} par une contrainte de validit\'{e} pour chaque ligne, colonne et petite grille (donc 27 clauses). 

La contrainte de validit\'{e} dit que l'ensemble sur lequel elle est appliqu\'{e} doit contenir exactement les chiffres de 1 \'{a} 9. En introduisant 9 variables bool\'{e}ennes par case, la validit\'{e} peut \^{e}tre formul\'{e} par des \textbf{definedness clauses}, assurant que chaque case contient une valeur de 1 \'{a} 9 ($\bigvee_{d=1}^{9}p_{ij}^d$, where $p_{ij}^d$ est vrai si $x_{ij} = d$) et des \textbf{uniqueness clauses}, assurant que chaque deux cases de l'ensemble consider\'{e} contiennent des valeurs differents (valid($x_1, x_2, \ldots, x_9 \Leftrightarrow \land_{1\leq i < j \leq 9} \land_{d=1}^{9} (x_i \neq d \vee x_j \neq d )$).

En total, on obtient 729 literaux (9 par case, 81 cases) et 11745 clauses, plus une clause par case remplie dans la grille initiale (qui fixe la valeur de la case).

On pourrait r\'{e}duire le nombre des clauses en ne considerant pas les contraintes de validit\'{e} pour les cases d\'{e}j\`{a} remplies, mais pour la claret\'{e} de la g\'{e}n\'{e}ration des clauses, on choisit de ne pas faire cette simplification.

\paragraph{Choix de structure:}

On choisit de repr\'{e}senter les formules par les types OCaml suivants:

\begin{itemize}
\item les indices d'une case par: \textbf{type cellule = \{ i : int; j : int \}}
\item les atomes par: \textbf{type atome = \{ cellule : cellule; d : int; signe : bool \}}; ainsi, $x_{ij}=d \Leftrightarrow$ {[\{cellule = \{i=i; j=j\}; d = d; signe = true\}]} et $x_{ij}\neq d \Leftrightarrow$ {[\{cellule = \{i=i; j=j\}; d = d; signe = false\}]}
\item une clause par une liste des atomes (qu'on disjonctionne)
\item une formule en forme normale conjoctive par une liste des clauses
\end{itemize}

\paragraph{Fichier en format dimacs}

Notre programme re\c{c}o\^{i} une grille Sudoku (en forme de ligne) et construit le fichier "dimacs.txt", qui contient la formule en forme normale conjoctive (fnc) qui lui correspond.

\begin{center}

\end{center}
\section{Minisat solver}
Si on appelle \textit{sudoku-solver} avec l'option \textbf{minisat}, le solveur minisat est utilis\'{e} sur le fichier contenant la fnc en format dimacs pour r\'{e}soudre la grille de Sudoku. 
\end{document} 


